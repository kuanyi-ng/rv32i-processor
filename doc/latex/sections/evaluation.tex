\documentclass[../main.tex]{subfiles}

\begin{document}
  設計したプロセッサの性能を, プログラム実行のクロックサイクル数, 
  最小動作クロック周期, 面積, と消費電力という面で評価する.

  \subsection{評価方法}
  MiBench ベンチマークプログラムを用いて, プログラム実行のクロックサイクル数を求めた.
  MiBench の中に, \verb|stringsearch, bitcnts, dijkstra| といったプログラムが含まれている.
  次に, 論理合成ツール Design Compiler を用いて, 
  論理合成を行い, 最小動作クロック周期, 面積, と消費電力を測定した.

  最小動作クロック周期の求め方について説明する.
  \begin{enumerate}
    \item タイミング制約を $10[ns]$ と設定して論理合成を行い, 
    論理合成の結果にある slack (与えたタイミング制約と最大遅延時間との差) が正であることを確認する.
    \item タイミング制約を $9[ns], 8[ns], \dots$ のように $1[ns]$ ずつ下げ, 
    slack が負になるタイミング制約を見つける.
  \end{enumerate}

  例えば, タイミング制約を $5[ns]$ に設定した時に slack が負になったら, 
  作成したプロセッサの最小動作クロック周期は $6[ns]$ になる.

  \subsection{評価結果}
  各ベンチマークのプログラムの実行に必要なクロックサイクル数を表 \ref{table:mibench-base} に示す.
  また, プロセッサの最小動作クロック周期, 面積, と消費電力の測定結果を表 \ref{table:logic-synthesis-base} に示す.

  \begin{table}[h]
    \centering
    \begin{tabular}{|c|r|}
    \hline
    ベンチマークプログラム  & クロックサイクル数 \\ \hline
    stringsearch & 10594     \\
    bitcnts      & 56040     \\
    dijkstra     & 4079473   \\ \hline
    \end{tabular}
    \caption{ベンチマークプログラムの実行クロックサイクル数}
    \label{table:mibench-base}
  \end{table}

  \begin{table}[h]
    \centering
    \begin{tabular}{|c|c|c|}
    \hline
    最小動作クロック周期 {[}$ns${]} & 面積 {[}$\mu m^2${]} & 消費電力 {[}$mW${]} \\ \hline
    6                   & 357534.722794    & 7.5732        \\ \hline
    \end{tabular}
    \caption{論理合成の結果}
    \label{table:logic-synthesis-base}
  \end{table}
\end{document}
