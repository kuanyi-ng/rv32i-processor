\documentclass[../main.tex]{subfiles}

\begin{document}
  設計したプロセッサの性能を, プログラム実行のクロックサイクル数, 
  最小動作クロック周期, 面積, ならびに, 消費電力で評価する.

  \subsection{評価方法}
  % どの大きさ(large, small, test)なのか、コンパイラの最適化レベルはいくつなのかという情報もあるといいかも。
  ベンチマークプログラム MiBench \cite{mibench} の一部を用い, プログラム実行のクロックサイクル数を求めた.
  今回の性能評価に使われたプログラムの一覧を付録\ref{appendix:programs}の表\ref{table:mibench-program} に示す.

  次に, 論理合成ツール Design Compiler を用いて論理合成を行い, 
  最小動作クロック周期, 面積, ならびに, 消費電力を測定した.

  最小動作クロック周期の求め方について説明する.
  \begin{enumerate}
    \item タイミング制約を $10ns$ と設定して論理合成を行い, 
    論理合成の結果にある slack (与えたタイミング制約と最大遅延時間との差) が正であることを確認する.
    \item タイミング制約を $9ns, 8ns, \ldots$ のように $1ns$ ずつ下げ, 
    slack が負になるタイミング制約を見つける.
  \end{enumerate}

  たとえば, タイミング制約を $5ns$ に設定した時に slack が負になったら, 
  作成したプロセッサの最小動作クロック周期は $6ns$ になる.

  \subsection{評価結果}
  各ベンチマークのプログラムの実行に必要なクロックサイクル数を表 \ref{table:mibench-base} に示す.
  また, プロセッサの最小動作クロック周期, 面積, ならびに, 消費電力の測定結果を表 \ref{table:logic-synthesis-base} に示す.

  \begin{table}[tbh]
    \centering
    \resizebox{\columnwidth}{!}{
      \begin{tabular}{|c|r|}
        \hline
        ベンチマークプログラム  & クロックサイクル数 \\ \hline
        stringsearch & 10594     \\
        bitcnts      & 56040     \\
        dijkstra     & 4079473   \\ \hline
      \end{tabular}
    }
    \caption{ベンチマークプログラムの実行クロックサイクル数(改善前)}
    \label{table:mibench-base}
  \end{table}

  \begin{table}[tbh]
    \centering
    \resizebox{\columnwidth}{!}{
      \begin{tabular}{|c|c|c|}
        \hline
        最小動作クロック周期 {[}$ns${]} & 面積 {[}$\mu m^2${]} & 消費電力 {[}$mW${]} \\ \hline
        6                   & 357534.7228    & 7.5732        \\ \hline
      \end{tabular}
    }
    \caption{論理合成の結果(改善前)}
    \label{table:logic-synthesis-base}
  \end{table}
\end{document}
