\documentclass[../main.tex]{subfiles}

\begin{document}

  \subsection{機能検証方法}
  用意されたプログラム(付録\ref{appendix:programs}の表\ref{table:test-program})
  を用いてプロセッサの機能検証を行う.
  機能検証は Verilog-HDL で記述したプロセッサに対し, 
  論理シミュレーター xmverilog と
  波形ツール SimVision を用いてシミュレーションを行った.
  テストプログラムが正しく実行され、正確な出力が得られたことを確認した.

  \subsection{性能評価方法}
  設計したプロセッサの性能を, プログラム実行のクロックサイクル数, 
  最小動作クロック周期, 面積, ならびに, 消費電力で評価する.

  ベンチマークプログラム MiBench \cite{mibench} の一部を用い, 
  コンパイラの最適化レベルをレベル3とした時のプログラム実行のクロックサイクル数を求めた.
  なお, ベンチマークプログラムが扱うデータのサイズが \verb|large|, \verb|small|, \verb|test| の3種類があるが, 
  評価に用いたのは \verb|test| サイズのプログラムである.
  今回の性能評価に使われたプログラムの一覧を付録\ref{appendix:programs}の表\ref{table:mibench-program} に示す.

  次に, 論理合成ツール Design Compiler を用いて論理合成を行い, 
  最小動作クロック周期, 面積, ならびに, 消費電力を測定した.

  最小動作クロック周期の求め方について説明する.
  \begin{enumerate}
    \item タイミング制約を $10.00\unit{ns}$ と設定して論理合成を行い, 
    論理合成の結果にある slack (与えたタイミング制約と最大遅延時間との差) が正であることを確認する.
    \item タイミング制約を $9.00\unit{ns}, 8.00\unit{ns}, \ldots$ のように $1.00\unit{ns}$ ずつ下げ, 
    slack が負になるタイミング制約を見つける.
  \end{enumerate}

  たとえば, タイミング制約を $5.00\unit{ns}$ に設定した時に slack が負になったら, 
  作成したプロセッサの最小動作クロック周期は $6.00\unit{ns}$ になる.

  \subsection{性能評価結果}
  各ベンチマークのプログラムの実行に必要なクロックサイクル数を表 \ref{table:mibench-base} に示す.
  また, プロセッサの最小動作クロック周期, 面積, ならびに, 消費電力の測定結果を表 \ref{table:logic-synthesis-base} に示す.

  \begin{table}[t]
    \centering
    \caption{ベンチマークプログラムの実行クロックサイクル数(改善前)}
    \label{table:mibench-base}
    \resizebox{\columnwidth}{!}{
      \begin{tabular}{|c|r|}
        \hline
        ベンチマークプログラム  & クロックサイクル数 \\ \hline
        stringsearch & 10594     \\
        bitcnts      & 56040     \\
        dijkstra     & 4079473   \\ \hline
      \end{tabular}
    }
  \end{table}

  \begin{table}[t]
    \centering
    \caption{論理合成の結果(改善前)}
    \label{table:logic-synthesis-base}
    \resizebox{\columnwidth}{!}{
      \begin{tabular}{|c|c|c|}
        \hline
        最小動作クロック周期 {[}\unit{ns}{]} & 面積 {[}$\unit{\um}^2${]} & 消費電力 {[}\unit{\mW}{]} \\ \hline
        6.00 & 357534.7228 & 7.5732 \\ \hline
      \end{tabular}
    }
  \end{table}
\end{document}
