\documentclass[../specifications]{subfiles}

\begin{document}
  パイプライン処理では、異なるステージにおいて、異なる命令が実行されている。
  2つの命令の間にデータ依存性が存在する時に、
  先に実行される命令が汎用レジスタを更新する前に、
  後で実行される命令が同じレジスタの古い値を読み出してしまうことがある。
  これによって、正確な演算結果を得ることができない。
  この現象を、RAW (Read After Write) ハザードという。
  以下、RAW ハザードが発生する場合を記述する。

  \begin{enumerate}
    \item 命令 $m$ はレジスタ \verb|xn| を更新し、
    命令 $m+1$ はレジスタ \verb|xn| の値を用いた演算を行う場合。

    \item 命令 $m$ はレジスタ \verb|xn| を更新し、
    命令 $m+2$ はレジスタ \verb|xn| の値を用いた演算を行う場合。
  \end{enumerate}

  RAW ハザードを解決するために、データフォワーディングとパイプラインストールの2つの方法がある。

  \subsubsection{データフォワーディング}
  データフォワーディングとは、EX (または、MEM) ステージにある命令 $m$ の演算結果を
  ID (または、EX) ステージにある命令 $m+1$ (または、命令 $m+2$) に渡し、
  命令 $m+1$ (または、命令 $m+2$) の EX (または、MEM) ステージで使用する方法である。
  この時に、命令 $m+1$ (または、命令 $m+2$) はレジスタ \verb|xn| の最新値
  % (命令 $m$ の WB ステージで更新される値)
  を用いて演算を行うため、正確な結果が得られる。

  データフォワーディングによって、上記の RAW ハザードが発生する場合の中で、
  命令 $m$ がロード命令以外の場合の対処ができる。
  データフォワーディングで解決できなかった場合は、
  パイプラインストールを用いる。

  \subsubsection{パイプラインストール}
  パイプラインストールとは、パイプラインの各ステージにある命令を
  次のステージに進まないようにする方法である。
  データフォワーディングに必要なデータが用意できるまでに、
  それ以降の命令を待たせる。
  
  パイプラインストールによって、
  データフォワーディングが解決できる状況が作られるため、
  RAW ハザードが解決できる。

  RAW ハザードが発生する状況とその解決法を表 \ref{table:data-hazard} にまとめる。

  \begin{table}[h!]
    \centering
    \begin{tabular}{ |c|c|c| }
      \hline
      レジスタ \verb|xn| を更新する命令 & レジスタ \verb|xn| を用いる命令 & 解決法  \\
      \hline
      ロード命令以外  & ストア命令    & \begin{tabular}[c]{@{}c@{}}データフォワーディング\\(EX/MEM → ID)\end{tabular}  \\
      \hline
      ロード命令以外  & ストア命令以外 & \begin{tabular}[c]{@{}c@{}}データフォワーディング\\(EX/MEM → ID)\end{tabular}  \\
      \hline
      ロード命令     & ストア命令    & \begin{tabular}[c]{@{}c@{}}データフォワーディング\\(MEM → EX)\end{tabular}  \\
      \hline
      ロード命令     & ストア命令以外 & パイプラインストール  \\
      \hline
    \end{tabular}
    \caption{RAW ハザードに関わる命令とその解決法}
    \label{table:data-hazard}
  \end{table}

\end{document}
