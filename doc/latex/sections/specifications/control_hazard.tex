\documentclass[../specifications.tex]{subfiles}

\begin{document}

  パイプライン処理が, 
  パイプライン処理をしない場合と異なる結果が得られる原因のもう1つは制御ハザードである.
  この章では, 分岐命令による制御ハザードについて説明してから, 
  今回のプロセッサの設計で採用した解決法について述べる.

  分岐命令が存在する時に, その命令の分岐結果によって次に実行する命令が決まる.
  パイプライン処理をしない場合, 分岐命令の実行を注意する必要はないが, 
  今回の5段パイプライン処理では, 分岐命令の分岐結果が EX ステージにならないと判明されない.
  それに, 分岐命令の後に続く命令がすでにパイプラインの IF ステージと ID ステージにおいて実行されている.
  そこで, 分岐結果が「分岐しない」ならば, 問題なく命令の実行を続くことができる.
  しかし, 分岐結果が「分岐する」ならば, IF ステージと ID ステージにある命令は実行してはいけない.
  IF ステージと ID ステージの命令を実行してしまったら, プログラムの実行結果が正しくなくなる.

  \subsubsection{パイプラインストール}
  パイプライン処理で, 分岐命令がある時に, 
  IF ステージと ID ステージにある命令を実行しない方法の1つとしてパイプラインストールがある.
  分岐命令の分岐結果が判明されるまでに, 分岐命令の後に続く命令がパイプラインに入らないように, 
  パイプラインをストールする方法である.
  そして, 分岐結果が分かってから, パイプラインストールを解除し, 
  分岐結果を元に選んだ次の命令をフェッチし, 分岐命令以降の命令の実行を再開する.
  しかし, パイプラインストールを採用すれば, 分岐命令がある度に,
  2つのクロックサイクルが無駄になってしまう.
  パイプラインストールよりクロックサイクルの無駄が少ない方法があるので, 
  パイプラインストールを採用しなかった.

  \subsubsection{パイプラインフラッシュ}
  パイプラインストールの他に, パイプラインフラッシュによって制御ハザードを解決する方法がある.
  分岐命令に続く命令がプロセッサの内部状態 (メモリと汎用レジスタ) に対する変更を無効化することを
  「パイプラインをフラッシュする」という.
  例えば, 分岐命令の分岐結果が「分岐する」であっても, IF ステージと ID ステージにある命令の実行を続ける.
  ただし, それらの命令がメモリと汎用レジスタに対する書き込みを無効化した状態の元で, 実行を続行する.
  その結果, 分岐結果が「分岐しない」場合では, クロックサイクルの無駄なく, 
  分岐命令に続く命令がそのまま実行されていく.
  しかし, 分岐結果が「分岐する」場合では, 2つのクロックサイクルが無駄になってしまう.
  パイプラインストールと比べたら, 分岐によるクロックサイクル数の無駄が少ないため, 
  今回の設計にパイプラインフラッシュを採用した.

\end{document}
