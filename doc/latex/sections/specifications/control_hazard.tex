\documentclass[../specifications.tex]{subfiles}

\begin{document}

  パイプライン処理が, 
  パイプライン処理をしない場合と異なる結果が得られる原因のもう1つは制御ハザードである.
  この節では, 分岐命令による制御ハザードについて説明した後に, 
  今回のプロセッサの設計で採用した解決法について述べる.

  分岐命令が存在する時に, その命令の分岐方向によって次に実行する命令が決まる.
  分岐方向とは, 分岐命令が分岐するか, 分岐しないかのことを意味する.
  パイプライン処理をしない場合, 分岐命令の実行を注意する必要はないが, 
  今回の5段パイプライン処理では, 分岐命令の分岐方向が EX ステージにならないと判明しない.
  その上, 分岐命令の後に続く命令がすでにパイプラインの IF ステージと ID ステージにおいて実行されている.
  分岐方向が「分岐しない」ならば, 問題なく命令の実行を続けることができる.
  しかしながら, 分岐方向が「分岐する」ならば, IF ステージと ID ステージにある命令は実行してはいけない.
  IF ステージと ID ステージの命令を実行してしまうと, データメモリ, または, 汎用レジスタが更新されることがある.
  それにより, プログラムの実行結果が正しくなくなる.

  \subsubsection{パイプラインフラッシュ}
  制御ハザードの解決法の1つとして, パイプラインストールがある.
  分岐命令の分岐方向が判明されるまでに, 分岐命令の後に続く命令がパイプラインに入らないように, 
  パイプラインをストールする方法である.
  そして, 分岐方向が判明した後, パイプラインストールを解除し, 
  分岐方向を基に次の命令をフェッチする.
  しかしながら, パイプラインストールを採用すると, 
  分岐命令が発生する度に2クロックサイクル分が無駄になってしまう.

  パイプラインストールの他に, パイプラインフラッシュによって制御ハザードを解決する方法がある.
  分岐命令に続く命令がプロセッサの内部状態 (メモリと汎用レジスタ) に対する変更を無効化することを
  「パイプラインをフラッシュする」という.
  たとえば, 分岐命令の分岐方向が「分岐する」場合でも, 
  内部状態の更新に関連する制御信号を無効化したまま, 
  IF ステージと ID ステージにある命令の実行を続ける.

  その結果, 分岐結果が「分岐する」場合では変わらず 2クロックサイクル分が無駄になるが, 
  分岐方向が「分岐しない」場合では, クロックサイクルの無駄が生じなくなる.
  パイプラインストールと比べると, 分岐によるクロックサイクル数のペナルティが少ないため, 
  今回の設計にパイプラインフラッシュを採用した.

\end{document}
