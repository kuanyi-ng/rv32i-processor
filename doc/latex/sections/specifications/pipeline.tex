\documentclass[../specifications.tex]{subfiles}

\begin{document}

  今回設計したプロセッサは, 1つの命令を5つのパイプラインステージに分けて実行する.
  それぞれのステージの名前と役割は以下の通りである.
  \begin{enumerate}
    \item IF ステージ
    \newline 次に実行する命令を命令メモリから読み出す.

    \item ID ステージ
    \newline 命令を解読して, EX ステージで行われる演算に必要な入力を用意する.

    \item EX ステージ
    \newline 命令で必要な演算を行う.

    \item MEM ステージ
    \newline データメモリへのアクセス (読み出し・書き込み) を行う.

    \item WB ステージ
    \newline 汎用レジスタへデータを書き込む.
  \end{enumerate}

  命令のパイプライン処理では, それぞれのステージにおいて, 
  どのように命令を実行するかを指定する制御信号が必要である.
  たとえば, EX ステージで ALU を使ってどんな演算を行うか, 
  MEM ステージでメモリにアクセスするかどうか, 
  WB ステージで汎用レジスタに値を書き込むかなどの制御信号がある.
  講義では, これらの制御信号を ID ステージで生成し, 
  パイプラインレジスタを介して後続のステージに伝播していく設計を学んだ.
  しかしながら, 後続のステージの処理までの制御信号の生成を考えることが難しかったため, 
  各々のステージに必要な制御信号は各ステージで生成するような設計にした.

  \end{document}
