\documentclass[../specifications.tex]{subfiles}

\begin{document}

  このプロセッサが対応している例外・割り込み処理の優先順位は以下の通りである.
  \begin{enumerate}
    \item リセット
    \newline 外部信号によってプロセッサの内部状態(レジスタとメモリ)をリセットする.

    \item 不正命令
    \newline サポートしていない命令を解読した時に例外処理に移す.

    \item 命令アクセス・ミスアライメント
    \newline 命令のアドレスは常に 4の倍数であるため, 
    命令メモリに対してそれ以外のアドレスにアクセスする時に例外処理に移す.

    \item \verb|ECALL| 命令
    \newline アプリケーション側がシステムコールを呼び出す時に \verb|ECALL| 命令を使う.
    システムコールの呼び出しを例外処理で行う.
  \end{enumerate}

  リセットがアクティブになる時, 
  プロセッサのメモリ, レジスタファイル, パイプラインレジスタと PC を初期値にリセットする.
  また, 不正命令, 命令アクセス・ミスアライメント, または, \verb|ECALL| 命令を検知した場合, 
  例外処理に移すために, PC の値を \verb|0x0000_0000| に設定し, 
  プログラマーが記述した例外処理のプログラムを実行する.
  なお, 例外処理前のプログラムに処理を移すために, 
  例外処理プログラムの最後に \verb|MRET| 命令を使用する必要がある.

\end{document}
