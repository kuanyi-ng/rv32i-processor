\documentclass[../improvements.tex]{subfiles}

\begin{document}

  分岐予測を実装した後と, クリティカルパスの短縮を行った後の性能評価の結果を
  表 \ref{table:mibench-improved} と表 \ref{table:logic-synthesis-improved}
  にまとめた.
  分岐予測の実装によるクロックサイクル数を実装前のと比較したら, 
  平均的に $79.4\%$ 減少したことがわかった.
  また, クリティカルパスの短縮により, 分岐予測の実装によって $9[ns]$ までに増えたクロック周期を
  $5[ns]$ へと減少させたことができた.

  \begin{table}[h]
    \centering
    \begin{tabular}{|c|r|r|}
    \hline
    ベンチマークプログラム  & クロックサイクル数 (分岐予測実装前) & クロックサイクル数 (分岐予測実装後) \\ \hline
    stringsearch & 10594               & 6966                                     \\
    bitcnts      & 56040               & 44680                                    \\
    dijkstra     & 4079473             & 3048011                                  \\ \hline
    \end{tabular}
    \caption{分岐予測実装前後のプログラム実行クロックサイクル数}
    \label{table:mibench-improved}
  \end{table}

  \begin{table}[h]
    \centering
    \begin{tabular}{|c|r|r|r|}
    \hline
    プロセッサ       & \multicolumn{1}{c|}{最小動作クロック周期 {[}ns{]}} & \multicolumn{1}{c|}{面積 {[}μm\textasciicircum{}2{]}} & \multicolumn{1}{c|}{消費電力 {[}mW{]}} \\ \hline
    分岐予測実装前         & 6                                        & 357534.7228                                       & 7.5732                             \\
    分岐予測実装後     & 9                                        & 936675.8334                                         & 10.6318                            \\
    クリティカルパス短縮後 & 5                                        & 764805.1213                                         & 15.3567                            \\ \hline
    \end{tabular}
    \caption{性能改善前後の論理合成の結果}
    \label{table:logic-synthesis-improved}
  \end{table}

\end{document}
