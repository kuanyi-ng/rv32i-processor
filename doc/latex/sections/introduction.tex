\documentclass[../main.tex]{subfiles}

\begin{document}

  現代のマイクロプロセッサの動作原理を理解することを目的とし, 
  基礎である命令パイプライン構造を持つプロセッサを設計・実装した.
  プロセッサの設計・実装は, Verilog-HDL というハードウェア記述言語を用いて行った.
  更に, プロセッサの最大遅延時間を削減するために, 
  分岐予測を導入して, クリティカルパスの短縮を試みた.
  分岐予測の導入によって, テストプログラムの実行クロックサイクル数が $20.59\%$ 減少した.
  クリティカルパスの短縮によって, 
  ボトルネックとなるステージの処理が他のステージに行われるようにすることで, 
  最大遅延時間が前より $33.33\%$ 減少した.

  本稿では, 第2章でプロセッサの仕様について述べる.
  第3章で, プロセッサの機能検証の方法について説明し, 
  第4章で, プロセッサの性能評価と論理合成の結果を示す.
  第5章で, プロセッサの性能改善方法として, 分岐予測とクリティカルパスの短縮について述べる.
  最後に, 第6章でまとめを行う.

\end{document}
