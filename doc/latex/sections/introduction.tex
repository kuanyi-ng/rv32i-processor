\documentclass[../main.tex]{subfiles}

\begin{document}

  マイクロプロセッサの動作原理を理解することを目的とし, 
  命令パイプライン構造を持つプロセッサを設計・実装した.
  プロセッサの設計・実装は, Verilog-HDL というハードウェア記述言語を用いて行った.
  さらに, プロセッサの最小動作クロック周期を削減するために, 
  分岐予測を導入して, クリティカルパスの短縮を試みた.
  分岐予測の導入により, テストプログラムの実行クロックサイクル数が平均的に $20.59\%$ 減少した.
  クリティカルパスの短縮により, 
  ボトルネックとなるステージの処理が他のステージに行われるようにすることで, 
  最小動作クロック周期が前より $33.33\%$ 減少した.

  本稿では, 第\ref{section:specifications}章でプロセッサの仕様について述べる.
  第\ref{section:testing}章で, プロセッサの機能検証の方法について説明し, 
  第\ref{section:evaluation}章で, プロセッサの性能評価と論理合成の結果を示す.
  第\ref{section:improvements}章で, プロセッサの性能改善方法として, 分岐予測とクリティカルパスの短縮について述べる.
  最後に, 第\ref{section:summary}章でまとめを行う.
  付録\ref{appendix:block-diagram}に設計したプロセッサの設計図(図\ref{fig:block-diagram})を載せている.
  プロセッサのソースコードは GitHub \footnotemark 上に置いてある.
  \footnotetext{https://github.com/kuanyi-ng/rv32i-processor}

\end{document}
