\documentclass[../main.tex]{subfiles}

\begin{document}

  今後, コンピュータアーキテクチャに関する研究を行う上で, 
  プロセッサの動作原理に対する理解とプロセッサの設計手法の会得が不可欠である.
  そこで, プロセッサの動作原理を理解することを目的とし, 
  命令パイプライン構造を持つプロセッサを設計・実装した.
  プロセッサはメモリから命令をフェッチして実行することを繰り返し, プログラムを実行する.
  また, 命令パイプライン構造を持つプロセッサは, 
  フェッチした命令を複数のステージに分けて実行するような構造を持つため, 
  1クロックサイクル内に複数の命令を並行して実行することが可能である \cite{ca-quantitative-approach}.

  プロセッサの設計・実装は, Verilog-HDL というハードウェア記述言語を用いて行った.
  さらに, プロセッサのクロックサイクル数と最小動作クロック周期を削減するために, 
  分岐予測の導入, ならびに, クリティカルパスの短縮を行った.
  分岐予測の導入により, テストプログラムの実行クロックサイクル数が平均で $20.59\%$ 減少した.
  また, ボトルネックとなるステージの処理の一部を他のステージで行い, 
  クリティカルパスの短縮をすることで, 
  最小動作クロック周期が短縮前に比べて $33.33\%$ 減少した.

  本稿では, 第\ref{section:specifications}章でプロセッサの仕様について述べる.
  第\ref{section:evaluation}章でプロセッサの機能検証の方法, ならびに, プロセッサの性能評価と論理合成の方法と結果を示す.
  第\ref{section:improvements}章でプロセッサの性能改善方法として, 分岐予測, ならびに, クリティカルパスの短縮について述べる.
  最後に第\ref{section:summary}章でまとめを行う.
  付録\ref{appendix:block-diagram}に設計したプロセッサの設計図(図\ref{fig:block-diagram})を載せている.
  プロセッサのソースコードは GitHub \footnotemark 上に置いてある.
  \footnotetext{https://github.com/kuanyi-ng/rv32i-processor}

\end{document}
