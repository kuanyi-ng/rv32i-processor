\documentclass[../main.tex]{subfiles}

\begin{document}

  現代のマイクロプロセッサの動作原理を理解することを目的とし, 
  基礎である命令パイプライン構造を持つプロセッサを設計・実装した.
  プロセッサの設計・実装は, Verilog-HDL というハードウェア記述言語を用いて行った.
  更に, プロセッサの最大遅延時間と面積を削減するために, 
  分岐予測を導入して, パイプラインの各ステージの役割分担を見直した.
  分岐予測の導入によって, テストプログラムの実行クロックサイクル数が (数字を入れる) 減少した.% TODO
  パイプラインの各ステージの役割分担の見直しによって, 
  ボトルネックとなるステージの処理が他のステージに行われるようにすることで, 
  最大遅延時間が (数字を入れる) 減少した. % TODO

  本稿では, 第2章でプロセッサの仕様について述べる.
  第3章で, プロセッサの機能検証の方法について説明し, 
  第4章で, プロセッサの性能評価と論理合成の結果を示す.
  第5章で, プロセッサの性能改善方法として, 分岐予測とクリティカルパスの短縮について述べる.
  最後に, 第6章でまとめを行う.

\end{document}
