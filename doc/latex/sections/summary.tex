\documentclass[../main.tex]{subfiles}

\begin{document}

  今回のプロセッサ設計演習では, 5段のパイプライン処理, 例外処理と分岐予測機能をもつプロセッサを設計した.
  テストプログラムを用いてプロセッサの動作確認を行い, プロセッサが正しく動作することを確認した.
  そして, MiBench ベンチマークプログラムを用いてプロセッサの性能評価を行い, 論理合成も行った.
  実際にプロセッサを設計するにあたり, 細かいところまで考慮しなければ, 思い通りの動作が出ないことを痛感した.
  この演習を通じて, 講義で習ったプロセッサの動作原理をより深く理解することができたと思う.
  また, 困難なことをやろうとする時に, それを複数個の実現可能なタスクに分解して行っていくことで, 
  モチベーションを保ちながら続けられることを学ぶことができた.

\end{document}
