\documentclass[../main.tex]{subfiles}

\begin{document}
  用意されたプログラム (表\ref{table:test-program}) を用いてプロセッサの機能検証を行う.
  機能検証は Verilog-HDL で記述したプロセッサに対し, 
  論理シミュレーター xmverilog と
  波形ツール SimVision を用いてシミュレーションを行った.
  テストプログラムが正しく実行され、正確な出力が得られたことを確認した.

  \begin{table*}[t]
    \centering
    \begin{tabular}{|c|c|l|}
    \hline
    テストプログラム & 言語 & プログラム内容 \\ \hline
    load & アセンブリ & ロード命令の動作検証 \\
    store & アセンブリ & ストア命令の動作検証 \\
    p2 & アセンブリ & 演算と分岐命令の動作検証 \\
    trap & アセンブリ & 命令の動作検証 \\
    hello & C & Hello World! をコンソールに表示するプログラム \\
    napier & C & Napier's Constant, $e$ の値を64桁の精度で計算するプログラム \\
    pi & C & pi の値を64桁の精度で計算するプログラム \\
    prime & C & 2を含め, 40個の素数を昇順に見つけるプログラム \\
    bubblesort & C & 100個の整数を Bubble Sort でソートするプログラム \\
    insertsort & C & 100個の整数を Insert Sort でソートするプログラム \\
    quicksort & C & 100個の整数を Quick Sort でソートするプログラム \\ \hline
    \end{tabular}
    \caption{機能検証用プログラム}
    \label{table:test-program}
  \end{table*}

\end{document}
